\documentclass[]{llncs}   % list options between brackets

\usepackage{color}
\usepackage{graphicx}
%% The amssymb package provides various useful mathematical symbols
\usepackage{amssymb}
%% The amsthm package provides extended theorem environments
%\usepackage{amsthm}
\usepackage{amsmath}

\usepackage{listings}

\usepackage{hyperref}

\usepackage{systeme}

\def\shownotes{1}
\def\notesinmargins{0}

\ifnum\shownotes=1
\ifnum\notesinmargins=1
\newcommand{\authnote}[2]{\marginpar{\parbox{\marginparwidth}{\tiny %
  \textsf{#1 {\textcolor{blue}{notes: #2}}}}}%
  \textcolor{blue}{\textbf{\dag}}}
\else
\newcommand{\authnote}[2]{
  \textsf{#1 \textcolor{blue}{: #2}}}
\fi
\else
\newcommand{\authnote}[2]{}
\fi

\newcommand{\knote}[1]{{\authnote{\textcolor{green}{Alex notes}}{#1}}}
\newcommand{\dnote}[1]{{\authnote{\textcolor{red}{Dima notes}}{#1}}}


% type user-defined commands here
\usepackage[T1]{fontenc}

\usepackage{xcolor}

\definecolor{dkgreen}{rgb}{0,0.6,0}
\definecolor{gray}{rgb}{0.5,0.5,0.5}
\definecolor{mauve}{rgb}{0.58,0,0.82}

\newcommand{\ma}{\mathcal{A}}
\newcommand{\mb}{\mathcal{B}}
\newcommand{\he}{\hat{e}}
\newcommand{\sr}{\stackrel}
\newcommand{\ra}{\rightarrow}
\newcommand{\la}{\leftarrow}
\newcommand{\state}{state}

\newcommand{\ignore}[1]{} 
\newcommand{\full}[1]{}
\newcommand{\notfull}[1]{#1}
\newcommand{\rand}{\stackrel{R}{\leftarrow}}
\newcommand{\mypar}[1]{\smallskip\noindent\textbf{#1.}\ \ \ }

\begin{document}

\title{On Space-Scarce Economy\\ In Blockchain Systems}
\author{Alexander Chepurnoy \and Dmitry Meshkov}
\institute{IOHK Research}
\maketitle

\begin{abstract}
In this paper we study space-scarce economy in massively replicated open blockchain systems. In these systems, such as Bitcoin, memory to hold a current state snapshot needed to validate transactions becomes the most scarce resource eventually. The issue is even more critical for blockchain systems used to store data~(votes, certificates, logs etc.). Uncontrolled state size growth could lead to security issues, such as denial-of-service attacks. Only technical solutions, not economic, have been proposed to tackle this problem to the moment. In contrast, we propose to add a new component to a transaction fee scheme based on how much additional space will be needed for new objects created in result of transaction processing and for how long they will live in the state. We provide three possible options towards implementing the new fee component, namely \textit{prepaid outputs}, \textit{postpaid outputs} and \textit{scheduled payments}. We provide an analysis of the model with respect to all the three options. We show that the state growth could be bounded by a fee factor, miners are getting additional stable rewards and lost coins are being taken back into circulation eventually.    
\end{abstract}

\section{Introduction}

Bitcoin~\cite{Nakamoto2008} was introduced in 2008 by S. Nakamoto as a purely peer-to-peer version of electronic cash with a ledger written into blockchain data structure securely replicated by each network node. Security of the scheme is relied on mining process. If majority of miners are honest, then Bitcoin meets its security goals as formal analysis~\cite{Garay2015} shows. For work done a miner is claiming a reward which consists of two parts. First, some constant number of bitcoins are created out of thin air according to a predefined and hard-coded token emission schedule. Second, a miner claims fees for all the transactions included into the block. A transaction fee is set by a user during transaction creation. Transaction fees are useful for an existing cryptocurrency economy for two reasons:

\begin{enumerate}
  \item{\em Incentivization of miners.} A rational Bitcoin miner does not include all the valid transactions into blocks as, due to the increased chances of orphaning a block, the cost of adding transactions to a block could not be ignored~\cite{andersen2013,rizun2015transaction}. As shown in~\cite{rizun2015transaction}, even in absence of block size limit, Bitcoin fee market is healthy and the miners surplus is maximized at a finite quantity of block space. Thus the miner is incentivized to produce a block of a limited size. This means that only a subset of transactions which provides enough value to a miner will be included in a block. A paper~\cite{rizun2015transaction} provides a procedure to calculate transaction fee based on block propagation time.

  \item{\em Limit resources usage and prevent spam. } Besides of network utilization, transaction processing requires a miner to spend some computational resources. For most of the cryptocurrencies, a transactional language is limited~(with Bitcoin Script~\cite{script} being one of the most limited), thus a number of CPU cycles needed to process a transaction is strictly bounded and corresponding computational costs are not directly considered. In contrast, in cryptocurrencies supporting smart contract languages, such as~\cite{seijas2016scripting,tezosScript,solidity}, transaction processing may require a lot of computations, and computational costs are included in transaction fee. This cost is specific to concrete transactional language and is out of scope of this paper.
\end{enumerate}

A transaction in Bitcoin fully spends outputs from previous transactions, and also creates new outputs of user-defined values. A notable and the only exception is a coinbase transaction of a block which creates fixed amount of money out of thin air and also claims transaction fees without referring to any outputs~(a fee for a non-coinbase transaction is sum of claimed outputs values minus sum of values for created outputs). A node is checking a transaction in Bitcoin by using a set of unspent outputs. In other cryptocurrencies a representation of a {\em state} needed to validate and process an arbitrary transaction could be different~(for example, in Ethereum~\cite{ethyp} such structure is called the {\em world state} and fixed by the protocol). To process a transaction quickly, the state~(or most accessed part of it) should reside in random-access memory. Once it becomes too big to fit into RAM an attacker can perform denial-of-service attacks against cryptocurrency nodes. For example, during attacks on Ethereum in Autumn, 2016, an attacker added about 18 million accounts to the state~(whose size was less than 1 million accounts before the attack) and then performed succesful denial-of-service attacks against the nodes\cite{eth2016dos}. Similarly, in 2013 a denial-of-service attack against serialized transactions residing in a secondary storage~(HDD or SSD) was discovered in Bitcoin\cite{vasek2014empirical}.

The main purpose of this paper is to consider a new mandatory component in a transaction fee scheme reflecting state growth. In all known cryptocurrencies of today, an element of the state once created lives possibly forever without paying anything for that. This leads to continuously increasing state~(we point to Bitcoin unspent transaction outputs~(UTXO) set size as an example~\cite{utxoChart}). Moreover, state may grow fast during spam attack, for example, 15 million outputs were quickly put into UTXO set during spam attacks against Bitcoin in July 2015~\cite{bitcoin2015flood}, and most of these outputs are not spent yet. The paper~\cite{reyzin2016improving} is proposing a technical solution for non-mining nodes where only miners hold the full state~(assuming that they can invest money in  random-access memory of sufficiently big capacity), while other nodes are checking proofs of state transformations generated by miners, and size of a proof (in average and also in a worst case) is about $log(S)$ in regards with a state size $S$. Nevertheless, big state could lead to centralization of mining or SPV mining~\cite{spvMining}, and these concerns should be addressed. Also, there is an increasing demand to use a blockchain as a data storage, and storing permanently objects in the state without a cleaning procedure is not a viable option.

We propose an economic solution to the problem of unreasonable state growth (such as spam attacks, or objects not being using anymore but still living in the blockchain). The solution is a new mandatory fee component. We state that a user should pay fee for both the additional space needed to store objects created by a transaction, and also for lifetime of new bytes. This model is usual for cloud storage services where users pay for gigabytes of data per month. We provide a possibility for miners to control their storage requirements by changing a fee factor. Later in this paper we will refer to this new fee component as to a {\em space-time fee}.

Proposed fee regime is promoting money circulation in the blockchain economy. The limited lifetime of a state element also leads to lost coins being taken back into circulation~(supposedly by miners). 

Summarizing, we study an economy where quick-access storage of a node in a massively-replicated system becomes the most scarce system resource eventually. Thus we call such an economy a {\em space-scarce economy}.

\subsection{Assumptions}
Here we provide assumptions our model is based on:

\begin{itemize}
  \item{} all the fees for a block are going to a single miner like in Bitcoin. There are proposals to share the rewards for a block within a group of miners, for example in~\cite{eyal2016bitcoin,kogias2016enhancing}, and they are out of scope of the paper.
  \item{} a state is a set of unspent outputs. An output is not modifiable so can be only created and then spent at whole. 
  \item{} an output is protected by a spending condition which is defined as a logical formula. Predicates in the formula can refer to properties of a blockchain (for example, its current height available via variable $Height$), spending transaction $tx$ and the output $out$ itself. We assume that it is possible to compare two scripts, and also it is possible to determine whether the spending transaction contains an output with a given property. For example, $tx.has\_output(script = out.script)$ evaluates to true if the spending transaction contains an output with the same script as the output has. Note that Bitcoin Script is too limited to support scripts comparison as well as using the spending transaction and the output to spend in a spending condition.

  \item{} for simplicity, we assume that a block is of a finite size but all the transactions a miner has at a moment of block generation can be packed into it, if otherwise is not stated explicitly.
  \item{} time is measured via \textit{height} which is a number of blocks since an initial block~(a genesis block) till a block of interest. 
  \item{} all anyone-can-spend outputs are collected by miners immediately as they appear.
  \item{} we are considering minimal mandatory fees in the paper. All the nodes are checking that a fee paid by a transaction is not less than a minimum and rejecting the whole block if it contains a transaction violating fee rules. Thus a fee regime is considered as a part of consensus protocol in our work. A user can pay more than the minimum to have a higher priority for a transaction of interest.   
\end{itemize}

\subsection{Structure of the Paper}

The paper is organized as follows. A design of our new fee component is provided in Section~\ref{sec:model}. The model then is analyzed in Section~\ref{sec:analysis}. In Section~\ref{sec:rel-work} we observe related work, and in Section~\ref{sec:further-work} we shape a plan for further research.

\section{The Model}
\label{sec:model}

In our model transaction fee $f$ consists of 3 parts: \textit{validation and processing cost} $f_v$, \textit{propagation cost} $f_p$ and \textit{state space-time difference cost} $f_s$, and $f_s$ is non-negative:

\begin{equation}
f = f_v + f_p + f_s , f \ge 0
\end{equation}

Validation and propagation costs are non-negative and assumed to be independent from the space-time fee. It is possible for $f_s$ to be negative since a transaction may decrease size of the state. The space-time fee $f_s$ depends on size of outputs to be marked as spent, remaining lifetimes for them, newly created outputs and their time to live. If a transaction is freeing some space-time \textit{$T_{cleared}$} and claiming space-time \textit{$T_{filled}$}, then the  resulting space-time fee can be calculated as follows:
\begin{equation}
f_s = K \cdot (T_{filled} - T_{cleared})
\end{equation}

\knote{we need to say somehow here about prepaid and postpaid outputs defined below}

where \textit{K} is a space-time price. As time in a blockchain system is associated with blocks~(precisely, height of a block), and space is measured in byte, $K$ is to be a price of byte per block. We assume that miners can change this price based on hardware prices and token exchange rate, like they change ``gas'' cost in Ethereum.

We consider three options for the fee to be charged. Concretely, \textit{prepaid outputs} are defined in Section~\ref{sec-prepaid}, \textit{postpaid outputs} are described in Section~\ref{sec-postpaid} and \textit{scheduled payments} are proposed in Section~\ref{sec-scheduled}.

\subsection{Prepaid Outputs}
\label{sec-prepaid}

First option we are going to analyze is prepaid outputs. In this model, an output of size \textit{B} is prepaid for a duration of $L_{max}$ blocks ($L_{max}$ is to be set explicitly in a transaction creating the output). After $L_{max}$ blocks after transaction inclusion into the blockchain, anyone~(presumably, a miner generating first block after the expiration) is able to spend the output. Thus a spending script for any output must be a combination of a regular user-defined spending script a condition on $L_{max}$:

\begin{equation}
(Height > out.height + L_{max}) \lor (regular\_script)
\end{equation}

where \textit{Height} is current blockchain height and {\em regular\_script} is user-defined spending script for the output. A mandatory minimal space-time fee to be paid by a transaction creating the output is following:

\begin{equation}
f_{s-prepaid} = K \cdot (\sum_i{(B_i \cdot (L_{max\_i} - Height))} - \sum_j{(B_j \cdot (L_{max\_j} - Height)}))
\end{equation}

\knote{Height is subjective here, write about that}

\subsection{Postpaid Outputs}
\label{sec-postpaid}

In the second model, a user pays fee for consumed space-time when he spends an output. When an output is spent its lifetime is known so resulting transaction fee can be calculated as:

\begin{equation}
f_{s-postpaid} = \sum_i{(K_i \cdot B_i \cdot (H_{spent\_i} - H_{creation\_i}))}
\end{equation}

where $H_{creation\_i}$ and $H_{spent\_i}$ are heights where i-th output was created and spent respectively. Note that each output has its own space-time price \textit{$K_i$} which is the \textit{K} value at the time of the output creation. We note that for output value \textit{V} a height \textit{$H_{max}$} exists when the whole output value will be consumed by the space-time fee:

\begin{equation}
H_{max} = {H_{creation} + {V \over {K \cdot B}}}
\end{equation}

After that anyone should be able to spend this output, and a spending script looks like

\begin{equation}
(Height > out.height + H_{max}) \lor (regular\_script)
\end{equation}


\subsection{Scheduled Payments}
\label{sec-scheduled}

The last model we introduce is a combination of the prepaid and postpaid models. Like in the postpaid model, a user pays for an output when he spends it in a transaction. In addition, the output is also enforced to be moved after a known duration $D_{s}$ since being touched last time. That is, if the user do not move the coins before this, then anyone~(presumably, a miner) can create a transaction claiming this output, returning all the coins back to the owner except of a space-time fee ${K \cdot B \cdot D_{s}}$. A spending script for the output would be like:

\begin{align}
\begin{split}
&(regular\_script) \lor \\
&(Height > (out.height + D_s) \land (out.value \le K \cdot B \cdot D_s \lor \\  
&\qquad tx.has\_output(value = out.value - K \cdot B \cdot D_s, script = out.script)))
\end{split}
\end{align}

In this model $D_s$ is another protocol parameter which may be fixed by a protocol design or changed via miners voting like \textit{K}.

\subsection{On K Parameter}
\label{sec-kparameter}

\knote{todo: choose proper title for the sub-section and fill the description}


\section{Analysis of the Model}
\label{sec:analysis}

In this section we provide analysis of the model. We show how the proposal is meeting its main goal which is the state size control. We get estimations for additional miners rewards as well as lost coins recirculation. Throughout this section we use following values got from the Bitcoin network at the moment of writing the paper~(February, 2017):

\begin{itemize}
  \item number of unspent transaction outputs: $N_{utxo} \approx 45,000,000$
  \item number of coins in circulation: $N_{coins} \approx 16,000,000$ BTC
  \item mean value of an output: $V_{mean} \approx 0.35$ BTC
  \item simple payment transaction fee: $f_{p} \approx 0.0002$ BTC
  \item mean output size: $B_{mean} \approx 36$ Bytes
  \item time interval between blocks: $T_{block} \approx 10$ minutes
  \item transaction fees per block: $M_{pg} \approx 1$ BTC
  \item constant block reward: $M_c = 12.5$ BTC
  \item average number of outputs spent in a block $N_{moved} \approx 5,500$ 
\end{itemize}

We use the numbers from the Bitcoin to make calculations for some typical cases as this is the most used cryptocurrency with the longest history known. However, a cryptocurrency with our proposal being deployed would have different values for its metrics. Thus the results from this section should be considered as initial estimations.

We study money flows in a space-scarce economy while dividing them into two types. Miners claim lost coins and also charge every output in the system state every block. Users are incentivized to move coins more frequently thus increasing mining rewards due to increased economic activity.  

\subsection{Lost coins recirculation}
\label{sec:re}

The need of lost coins recirculation has been discussed in the literature~\cite{gjermundrod2014recirculating,gjermundrod2016going} in regards with combat deflation that will eventually occur in cryptocurrencies with fixed supply. Possible cryptocurrency supply is known by design but actual supply is not known due to lost keys~\cite{ron2013quantitative}. Deflation is a problem for traditional monetary systems, however, it is still an open question whether it will be problematic for the Bitcoin economy~\cite{bitcoinDeflationarySpiral,barber2012bitter}.


Our model provides lost coins recirculation mechanism by design. In this section we study how recirculation depends from charging approach chosen as well as parameters.

Fee for keeping output of size \textit{B} in the \state{} for \textit{L} blocks is ${K \cdot B \cdot L}$. In \textit{Postpaid} and \textit{Scheduled payments} models space-time fee is taken from output value itself so when keys for an output which value is \textit{$V_{output}$} are lost the coins will be taken back to circulation after \textit{$L_{max}$} blocks: 

\begin{equation}
{L_{max} = {V_{output} \over K \cdot B}}
\end{equation}

Maximum lifetime of a concrete output depends on the miner-controlled \textit{K} parameter and user-defined output size and value. For a typical case in Bitcoin:

\begin{equation}
{L_{max} \approx {1 \over 100 \cdot K}}
\end{equation}

From this equation it is possible to calculate \textit{K} by specifying target expected lifetime for an output of an average size. For example, in case of targeted 50 years period of recirculation for an average output, value of \textit{K} parameter would be about $10^{-9}$. We will use this estimated $K$ value further in the analysis. Significant difference between \textit{Postpaid} and \textit{Scheduled payments} is that in \textit{Postpaid} lost output will return to economy all at once after potentially big $L_{max}$ period, while in the \textit{Scheduled payments} mode they will return to economy by portions during this period. This makes \textit{Scheduled payments} preferable to \textit{Postpaid} in regards with mining rewards stability. Anyway coins recirculation rate for both the modes in a long-term would be calculated as

\begin{equation}
R_{postpaid} = R_{scheduled} = {N_{lost} \cdot V_{output} \over L_{max}} = {N_{lost} \cdot K \cdot B}
\end{equation}

where $N_{lost}$ is the number of outputs with lost keys. Assuming current Bitcoins statistics, $N_{lost}\approx10^6$ and $K = 10^{-9}$ we can estimate that $\approx0.036$ bitcoins are to be released from abandoned outputs in every block.

In the \textit{Prepaid} mode a user defines $L_{max}$ by himself when he creates an output and number of released coins after $L_{max}$ is equals to output value $V_{output}$ rather than much smaller space-time fee $f_s$. Note that $L_{max}$ do not directly depends on \textit{K} and coins recirculation rate will only depend on $V_{output}$, user-defined $L_{max}$ and number of lost outputs $N_{lost}$:

\begin{equation}
R_{prepaid} = {N_{lost} \cdot V_{output} \over L_{max}}
\end{equation}

Assuming that $K = 10^{-9}$ we can estimate difference between prepaid and postpaid models:

\begin{equation}
{R_{prepaid} \over R_{postpaid}} = {V_{output} \over  K \cdot B \cdot L_{max}} \approx {10^7 \over L_{max}}
\end{equation}

For example, if user-defined coin recirculation period $L_{max} = 10$ years, then ${R_{prepaid} \over R_{postpaid}} \approx 19$

Concluding, our fee model by its design provides a way to return lost coins to circulation thus preventing potential deflation. \textit{Postpaid} model has a disadvantage  that lost coins could be returned to circulation in far future. In \textit{Prepaid} model coin recirculation rate is presumably much higher than in other models and \textit{Scheduled payments} mode is the most smooth way to recycle coins.



\subsection{User-Driven Money flow}
\label{sec:flow}

From users point of view, the longer you keep some output, the bigger space-time fee you pay. Thus the proposed fee regime provides an incentive to move coins more frequently stimulating economic activity. In all the fee models introduced in Section~\ref{sec:model}, a user pays ${K \cdot B}$ coins every block for keeping his output in the state. 

Currently in Bitcoin $N_{moved}$ outputs are moved in a block, and this number can be defined as: 

\begin{equation}
N_{moved} = {N_{utxo} \over L_{mu}}
\end{equation}
where $L_{mu}$ is mean lifetime of an output. We can then define $L_{mu}$ as:

\begin{equation}
L_{mu} = {N_{utxo} \over {N_{moved}}}
\end{equation}

In Bitcoin, $L_{mu} \approx 8182$ blocks.


We expect users moving coins more frequently thus mean lifetime becomes $L_{mi} < L_{mu}$. $L_{mi}$ depends on K and also on blockchain-specific usage scenarios. Increasement in money flow could be calcualted as:

$\Delta N_{moved} = {N_{utxo} \over L_{mi}} - {N_{utxo} \over L_{mu}} = N_{utxo} \cdot {{L_{mu}-L_{mi}} \over {L_{mu} \cdot L_{mi}}}$



\subsection{Miner rewards}
\label{minerrew}
Incentivizing mining is very important for viability of a cryptocurrency economy. 
In addition to usual propagation fee, the new fee regime provides two new sources of miner's income: a space-time fee and lost coins in the \textit{Prepaid} model.

Thus in the \textit{Prepaid} model:
\begin{equation}
M_{3} = M_{propagation} + M_{space-time} + M_{lost}
\end{equation}

In other models:
\begin{equation}
M_{2} = M_{propagation} + M_{space-time}
\end{equation}

We now calculate the difference with reward $M_{pg}$~(propagation fees) miners have in Bitcoin. We assume that number of inputs per transaction is the same, then number of transactions as well propagation fees are to be increased in the same way as $N_{moved}$:

\begin{equation}
M_{propagation} = M_{pg} \cdot(1 + {{L_{mu}-L_{mi}} \over {L_{mu} \cdot L_{mi}}})
\end{equation}


A space-time reward depends on state size and space-time price $K$ only:

\begin{equation}
M_{space-time} = \sum_{i=1}^{N_{utxo}} K_i \cdot B_i \approx K \cdot N_{utxo} \cdot B
\end{equation}

Recirculation reward in the \textit{Prepaid} model:

\begin{equation}
M_{lost} = {N_{lost} \cdot V_{output} \over L_{max}}
\end{equation}

Thus the additional mining reward is:

\begin{equation}
\Delta M_2 = M_2 - M_{pg} = M_{pg} \cdot {{L_{mu}-L_{mi}} \over {L_{mu} \cdot L_{mi}}} + K \cdot N_{utxo} \cdot B
\end{equation}

\begin{equation}
\Delta M_3 = M_{pg} \cdot {{L_{mu}-L_{mi}} \over {L_{mu} \cdot L_{mi}}} + K \cdot N_{utxo} \cdot B + {N_{lost} \cdot V_{output} \over L_{max}}
\end{equation}

We remark that space-time reward $K \cdot N_{utxo} \cdot B$ does not depend on network throughput and can be easily predicted. thus it can be considered as a constant block reward making mining more stable and thus securing the network. The value of it may be controlled by adjusting $K$ parameter.

\subsection{State size}

In this section we analyze the main goal of the proposed fee model which is to control size of the state. We study two most important cases. First, we estimate how big the state should be to give miners rewards comparable to Bitcoin's constant block reward of today~(12.5 bitcoins per block at the moment of writing this paper). Second, we study what is the maximum size of state could be considering an attacker having all the coins in the system initially and performing constant spam attack.

\paragraph{Constant Rewards}

For a state of size $S$ miner is getting $S \cdot K$ space-time fees every block. For them to be a replacement for the current constant reward in Bitcoin $M_c$ following equality should hold:

\begin{equation}
\label{eq:statesizeexp}
S \cdot K = M_c
\end{equation}
which leads to 13 GB state for $K=10^{-9}$ $\frac{BTC}{Byte \cdot Block}$.


\paragraph{Spam Attack}

We consider an attacker who creates an output roughly of size of a block $S_{block}$ for every block in a row. Initially the attacker has all the $N_{coins}$ of the system and he does not buy new coins during the attack~(miners do not sell the coins to him). We also assume that $K$ is constant during the attack~(however, in real world miners would probably raise $K$ during the attack). As soon as each block space-time fee should be paid for all previously created outputs, maximum possible number of such outputs $O_{max}$ is to be calculated from equation:

\begin{equation}
N_{coins} = \sum_i^{O_{max}}{K \cdot S_{block} \cdot i} \approx K \cdot S_{block} \cdot O_{max}^2 / 2
\end{equation}
assuming that number of unspent outputs is big enough. Resulting state size: 

\begin{equation}
\label{eq:statesizemax}
S = S_{block} \cdot O_{max} = \sqrt { 2 \cdot N_{coins} \cdot S_{block} \over K}
\end{equation}

which is about 183 gigabytes for Bitcoin network parameters and $K=10^{-9}$ $\frac{BTC}{Byte \cdot Block}$. %Taking into account that \state{} should be kept in memory, 183GB is quite big state size, however, miners can keep it in RAM. 
Note that it is upper-bound of \state{} size calculated from non-realistic assumptions that all available supply will be paid for space-time fee, and $K$ is constant.

Summarizing, the new fee regime allows to control state growth with $K$ parameter and allows to estimate expected state size as well as an upper bound. Reasonable $K$ value for a Bitcoin-like system leads to 13 Gb of expected state size and 183 GB for a spam attack backed by all the system tokens. Thus the model is preventing spam attacks like~\cite{bitcoin2015flood} and provide a guarantee that miners can have state fit in RAM.

\section{Related Work}
\label{sec:rel-work}

Ethereum~\cite{ethyp} introduced a fee to cover computational costs for processing a transaction~(via rewarding a miner), with a mandatory minimum. A price of a computational unit~(which is an instruction of Ethereum Virtual Machine) is controlled by miners via adjusting ``gas'' price. We consider to adopt this approach toward charging users for consuming space-time units.

FreiCoin~\cite{freicoin} is a cryptocurrency with demurrage. Demurrage-adjusted running balance~(aggregated for all the outputs associated with a key) is to be recalculated when a transaction is touching any output which is a part of the balance. Unlike FreiCoin, miners are watching over the state constantly to clean it in our proposal, and coins being lost are getting back into circulation~(in FreiCoin, outputs forgotten do live forever in the unspent outputs set).

In PascalCoin~\cite{pascalCoin}, if an account has not made any operations for 420,480 blocks then its balance is considered as forgotten and miner can claim it. Our approach to recover lost coins is more flexible and natural. %In the \textit{prepaid} mode a user sets expiration time on his side~(paying accordingly), in \textit{postpaid} and \textit{scheduled payments} modes a forgotten output is to be turned into space-time fees eventually.

\section{Further Work}
\label{sec:further-work}

We left some questions for further research:

\begin{itemize}
  \item{}it is interesting to see how tokens in a cryptocurrency with a proposed fee component can be compared with a Gesell's demurrage currency~\cite{gesell1958natural}.
  \item{}an interesting research vector is how to find appropriate mandatory minimal $f_v$, $f_p$, $f_s$ values. Another question is how these component could be combined in a compound minimal transaction fee $f$. aside of a simple sum, one possible option is to set $f = max(f_p, f_v, f_s)$.
  \item{}for currency blockchains, a subsidized period of free storage could be applied. One possible option is discussed in Appendix~\ref{sec:subsidy}.
  \item{}a possible interaction with the problem of blockchain instability without the constant block reward~\cite{carlsten2016instability} in the setting of increased fee supply is another concern to study.
\end{itemize}

\bibliographystyle{elsarticle-num}
\bibliography{sources.bib}

\appendix

\section{On Subsidizing Space-time}
\label{sec:subsidy}

It is usual to pay space-time fee for a cloud storage and thus our new fee model is natural for a blockchain acting as a database. On the other hand, in a cryptocurrency setting it may be natural to allow users to keep their coins for free for some period of time $S_s/B$. For example $S_s \approx 1.8$ ($Mb \cdot Block$) will allow to keep output for free for one year assuming 10 minutes block interval. While it is quite a big period for regular output, it is negligible for blockchain usage as a storage, e.g. will only allow to keep 12 Kb output for one day in \state{}.

For sure, this free space time will increase \state{} size upper bound. To maximize attack damage, the adversary will maximize number of created outputs, getting $S_s$ space-time for each for free, filling all blocks during subsidized period $S_s/B$. In case of Bitcoin parameters one year of subsidized period for smallest outputs will make it possible to keep 51 Gb of outputs for free, increasing \state{} size upper bound for 27\%.

Thus, additional subsidized space-time parameter allows to use the same blockchain for both money and data applications. Also this additional parameter provide a new way to configure cryptocurrency, for example to increase $K$ value painlessly for regular users, but increasing lost coin recirculation rate.

\section{Unified Transactions}

\knote{We can throw away special coinbase transactions getting unified transaction format, by requiring fees to be paid in form of anyone-can-spend output, and by creating constant rewards in predefined and hard-coded pre-genesis state, where a reward for a certain height is an output in the pre-genesis UTXO spendable only at this height.}

\section{Questions and Answers}


\paragraph{\textbf{Q: } Aren't there important use-cases for ``storing data forever''? Like when the ledger is used for storing property deeds or university degrees or other public, "official" documents?}~\\
\textbf{A: } While this is very appealing for a user to store something forever paying just once~(or not paying anything at all), a resulting economy where miners need to store increasingly growing amount of data getting bounded rewards may be unstable or just not viable. As a solution, we propose a system where a state size could be bounded, then, as new data is to be introduced each block, expiration is strictly needed. However, for some real-life scenarios we can imagine hybrid systems, where some special actors~(governments or universities) can store something forever~(paying once, or not paying at all), while other participants are paying as we are proposing thus subsidizing public good.

\paragraph{\textbf{Q: } If the miner can fully control K, how do you prevent a miner from setting K VERY high? Should not there be some sort of guarantee about the minimum value of $L_{max}$ BEFORE a transaction is made?}~\\
\textbf{A: } We assume miners will behave rationally, so they do not want to kill the business by setting $K$ high. We propose to have a rule for changing $K$ slowly, as block gas limit in Ethereum~(say, a miner can increase or decrease $K$ by 0.1\% per block) so $L_max$ could be estimated before the transaction being made with a good precision.
\end{document}