\documentclass[]{llncs}   % list options between brackets

\usepackage{color}
\usepackage{graphicx}
\graphicspath{{./figures/}}
\usepackage{subcaption}
%% The amssymb package provides various useful mathematical symbols
\usepackage{amssymb}
\usepackage{standalone}
\usepackage{pgfplots}
%% The amsthm package provides extended theorem environments
%\usepackage{amsthm}
\usepackage{amsmath}
\usepackage{tikz}

\usepackage{listings}

\usepackage{hyperref}

\usepackage{systeme}

\usepackage{enumitem}

\usepackage{float}

\def\shownotes{1}
\def\notesinmargins{0}

\ifnum\shownotes=1
\ifnum\notesinmargins=1
\newcommand{\authnote}[2]{\marginpar{\parbox{\marginparwidth}{\tiny %
  \textsf{#1 {\textcolor{blue}{notes: #2}}}}}%
  \textcolor{blue}{\textbf{\dag}}}
\else
\newcommand{\authnote}[2]{
  \textsf{#1\textcolor{blue}{ #2}}}
\fi
\else
\newcommand{\authnote}[2]{}
\fi

\newcommand{\knote}[1]{{\authnote{\textcolor{green}{Alex notes:}}{#1}}}
\newcommand{\dnote}[1]{{\authnote{\textcolor{red}{Dima notes:}}{#1}}}
\newcommand{\vk}[1]{{\authnote{\textcolor{red}{V:}}{#1}}}

% type user-defined commands here
\usepackage[T1]{fontenc}

\usepackage{xcolor}

\definecolor{dkgreen}{rgb}{0,0.6,0}
\definecolor{gray}{rgb}{0.5,0.5,0.5}
\definecolor{mauve}{rgb}{0.58,0,0.82}

\pgfplotsset{compat=newest, table/search path=figures}

\begin{document}

\title{A Systematic Approach To Cryptocurrency Fees}

\author{Alexander Chepurnoy\inst{1,2}, Vasily Kharin\inst{3}, Dmitry Meshkov\inst{1,2}}

\institute{Ergo Platform \and
IOHK Research \\\email{\{alex.chepurnoy, dmitry.meshkov\}@iohk.io} \and
Helmholtz Institute Jena, Froebelstieg 3, 07743, Jena, Germany \\\email{v.kharin@protonmail.com}}

\maketitle

\begin{abstract}
%In this paper we study transaction fees in massively replicated open blockchain
%systems. In these systems, such as Bitcoin, memory to hold a current state
%snapshot needed to validate transactions becomes a scarce resource eventually.
%Uncontrolled state size growth could lead to security issues.  We propose to add
%a new component to a transaction fee, which is based on how much additional
%space will be needed for new objects created as a result of transaction
%processing and for how long will they live in the state.  We propose how to
%combine fees charged for different resources spent~(bandwidth, random-access
%state memory, processor cycles) in a composite fee and argue that such
%separation is reasonable, based on statistics from Ethereum network.  We show a
%possible implementation for state-related fee in a form of regular payments to
%miners.

This paper is devoted to the study of transaction fees in massively replicated
open blockchain systems. In such systems, like Bitcoin, a snapshot of current
state required for the validation of transactions is being held in the memory,
which eventually becomes a scarce resource.  Uncontrolled state growth can lead
to security issues.  We propose a modification of a transaction fee scheme
based on how much additional space will be needed for the objects created as a
result of transaction processing and for how long will they live in the state.
We also work out the way to combine fees charged for different resources
spent~(bandwidth, random-access state memory, processor cycles) in a composite
fee and demonstrate consistency of the approach by analyzing the statistics from
Ethereum network. We show a possible implementation for state-related fee in a
form of regular payments to miners.

\end{abstract}

\section{Introduction}

Bitcoin~\cite{Nakamoto2008} was introduced in 2008 by S. Nakamoto as a purely
peer-to-peer version of electronic cash with a ledger written into blockchain
data structure securely replicated by each network node. Security of the cryptocurrency 
relies on its mining process. If majority of miners are honest, then Bitcoin
meets its security goals as formal analysis~\cite{Garay2015} shows. For the work
done a miner is claiming a reward which consists of two parts. First, some
constant number of bitcoins are created out of thin air according to a
predefined and hard-coded token emission schedule. Second, a miner claims fees
for all the transactions included into the block.

As shown in~\cite{carlsten2016instability} constant block rewards are an important
part of the Bitcoin protocol. Once a predetermined number of coins will enter the circulation
and miners will be rewarded by transaction fees only, their rational behavior will
dramatically differ from the default mining strategy. It is still an open question, whether
Bitcoin will meet its security goals in such circumstances, but at least number of
orphaned blocks will increase making Bitcoin less friendly for regular users.

A transaction fee, which is set by a user during transaction creation, is
useful to limit miners resource usage and prevent spam. In most cases a user pays a fee proportional to transaction size,
limiting miners {\em network} utilization. A rational miner does not
include all the valid transactions into blocks as, due to the increased
chances of orphaning a block, the cost of adding transactions to a block
could not be ignored~\cite{andersen2013,rizun2015transaction}. As shown
in~\cite{rizun2015transaction}, even in the absence of block size limit
Bitcoin fee market is healthy and the miner's surplus is maximized at a
finite size of a block. Thus miners are incentivized to produce blocks of a limited size, so 
only transactions providing enough value to a miner will be included in a block. The 
paper~\cite{rizun2015transaction} provides a procedure to estimate
transaction fee based on block propagation time.

Besides network utilization, transaction processing requires a miner
to spend some {\em computational} resources.
In Bitcoin the transactional language\cite{script} is very limited, and 
a number of CPU cycles needed to process a transaction
is strictly bounded, and corresponding computational costs are not included in the fee.
In contrast, in cryptocurrencies supporting smart contract
languages, such as Solidity~\cite{solidity} and Michelson~\cite{tezosScript},
transaction processing may require a lot of computations, and
corresponding costs are included in the transaction fee. Analysis of this fee component is done for concrete systems
in \cite{Earlz2017,luu2015demystifying}, and is out of scope of this paper.

In this work we address the problem of miners {\em storage} resources utilization.
A regular transaction in Bitcoin fully spends outputs from previous transactions, and
also creates new outputs of user-defined values and protecting scripts.
%A notable and the only
%exception is a coinbase transaction of a block which creates fixed amount of
%money out of thin air and also claims transaction fees without referring to any
%outputs~(a fee for a non-coinbase transaction is sum of claimed outputs values
%minus sum of values for created outputs).
A node checks a transaction in
Bitcoin by using a set of unspent outputs~(UTXO). In other cryptocurrencies a
representation of a {\em state} needed to validate and process an arbitrary
transaction could be different~(for example, in Ethereum~\cite{ethyp} such
structure is called the {\em world state} and fixed by the protocol). To process
a transaction quickly, the state~(or most accessed part of it) should reside in
expensive random-access memory~(RAM). Once it becomes too big to fit into RAM an attacker can
perform denial-of-service attacks against cryptocurrency nodes. For example,
during attacks on Ethereum in Autumn, 2016, an attacker added about 18 million
accounts to the state~(whose size was less than 1 million accounts before the
attack) and then performed successful denial-of-service attacks against the
nodes~\cite{eth2016dos}. Similarly, in 2013 a denial-of-service attack against
serialized transactions residing in a secondary storage~(HDD or SSD) was
discovered in Bitcoin~\cite{vasek2014empirical}.

In all the cryptocurrencies we are aware of, an element of the state once created
lives potentially forever without paying anything for that. This leads to perpetually increasing
state~(e.g. the Bitcoin UTXO size~\cite{utxoChart}). Moreover, state may grow fast during spam attacks, for
example, 15 million outputs were quickly put into the UTXO set during spam attacks
against Bitcoin in July, 2015~\cite{bitcoin2015flood}, and most of these outputs
are not spent yet. The paper~\cite{reyzin2016improving} is proposing a technical
solution for non-mining nodes where only miners hold the full state~(assuming
that they can invest money in random-access memory of sufficiently large
capacity), while other nodes are checking proofs of state transformations
generated by miners, and a size of a proof (in average and also in a worst case)
is about $log(|s|)$, where $|s|$ is a state size. Nevertheless, big state
could lead to centralization of mining or SPV mining~\cite{spvMining}, and these
concerns should be addressed. The question of internalizing the costs of
state load was raised in~\cite{Moeser2015}, but to the best of our knowledge there has not been any
practical solution proposed yet. Also, there is an increasing demand to use a
blockchain as a data provider, and permanently storing objects in the state
without a cleaning procedure in such a case is not a viable option.

\subsection{Our contribution}

In this paper we propose an economic solution to the problem of unreasonable state growth
(such as spam attacks or objects not being used anymore but still living in
the validation state). It consists in introducing a new mandatory fee component. A 
user should pay a fee based on both the additional space needed to store objects
created by a transaction, and the lifetime of the new bytes. Such an approach is
typical for the cloud storage services where users pay for gigabytes of data per
month. 

We also consider a method of combining fees for various resources consumed by a transaction:
bandwidth, random-access memory to hold state, and processor cycles to process computations prescribed by the transaction.    
The option being analyzed is to charge only for a resource which is consumed most of all, so we can talk about storage-oriented,
network-oriented or computation-oriented transactions. The evaluation is
conducted for Ethereum usage data, and it shows
that it is both possible and meaningful for this cryptocurrency to determine transaction type.

A way to charge for state memory consumption~(with the
output lifetime taken into account) is proposed as well.
Our scheme of ``scheduled payments'' is convenient for users not knowing the
duration of their outputs' storage in advance.


%We provide a possibility for miners to control their storage requirements by changing a fee factor. 
%Later in this paper we will refer to this new fee component as to a {\em space-time fee}.

%Proposed fee regime is promoting money circulation in the blockchain economy.
%The limited lifetime of a state element also leads to lost coins being taken
%back into circulation~(supposedly by miners). 

%Summarizing, we study an economy where quick-access storage of a node in a
%massively-replicated system becomes the most scarce system resource eventually.
%Thus we call such an economy a {\em space-scarce economy}.

\subsection{Structure of the paper}
The paper is organized as follows. The model assumptions and their analysis are in
Section~\ref{sec:preliminaries}. An algorithm for a composite fee assignment is
in Section~\ref{sec:algorithm}. A possible approach to charging for state memory
consumption is in Section~\ref{sec:scheduled}. The results of Ethereum data
evaluation are in Section~\ref{sec:evaluation}. Section~\ref{sec:conslusion}
contains the conclusions.

%A design of our new fee component is
%provided in Section~\ref{sec:model}. The model then is analyzed in
%Section~\ref{sec:analysis}. In Section~\ref{sec:rel-work} we observe related
%work, and in Section~\ref{sec:further-work} we shape a plan for further
%research.

\section{Preliminaries}
\label{sec:preliminaries}

We shape our model with the following assumptions:
\begin{itemize}%[label=\textbf{A\arabic*. }]
  % \item All the fees for a block are going to just a miner like in Bitcoin.
  %    There are proposals to share the rewards for a block within a group of
  %    miners, for example in~\cite{eyal2016bitcoin,kogias2016enhancing}, and
  %    they are out of scope of the paper. A notable difference is that if a single miner
  %    is taking all the rewards, he can include his own transactions in it for free
  \item \label{a:utxo} a transaction creates new objects called outputs and spends outputs from previous transactions. Thus the state needed for transaction validation consists of an unspent outputs. The size of the state then is the sum of sizes of all the unspent outputs.
  \item \label{a:state} single transaction does not change size of the state significantly
  \item \label{a:miner} it is profitable for a miner to collect fees from unspent outputs. 
  %\item For simplicity, we assume that a block is of a finite size but all the
  %    transactions a miner has at a moment of block generation can be packed
  %    into it, if otherwise is not stated explicitly
  \item \label{a:minimal} we are considering minimal mandatory fees in the paper. All the nodes
      are checking that a fee paid by a transaction is not less than a minimum
      and rejecting the whole block if it contains a transaction violating fee
      rules. Thus a fee regime is considered as a part of consensus protocol in
      our work. A user can pay more than the minimum to have a higher priority
      for a transaction of interest to be included into a block.
\end{itemize}


\section{An algorithm for the fee assignment}
\label{sec:algorithm}

<<<<<<< HEAD
As mentioned in the introduction, we develop a fee regime having two goals
in mind, namely incentivization of miners and spam prevention.  In this chapter
we reason about the guiding  principles for the fee assignment, and end up with
an example of a practically useful fee assignment rule.

The evolution of blockchain networks has demonstrated the main resources
being used. First and the most important so far, the memory of network nodes 
is limited resource. Blocks in the blockchain after processing are stored in a 
secondary storage, where a cost of a storage unit is low. In contrast, to validate a 
transaction, some state is needed~(for example, unspent outputs set in Bitcoin is used 
to validate a transaction), and this state should reside in expensive random-access memory.   

Second, it is obvious, especially with the development of smart contracts,
that a cost to process a transaction can be more than just a storage cost:
transactions can contain relatively complicated scripts which are meant to be
executed by all the nodes in the network. The most famous example is
the Ethereum network implementing the concept of a ``world computer''~\cite{ethyp}. 

Third, there is the network load created by every transaction. If an output is
created in one block and spent right in the next one, it provides almost zero
overhead in terms of validation state size, but creates the network load needed
for synchronization.

A transaction fee should incorporate all the three components stated above.  As 
shown in~\cite{Earlz2017}, assigning the fee to the storage as if
it was execution of some code can lead to significant disbalance for rich enough
scripting language (for example, for the data being written with an opcode other
than the conventional storage one). Thus, we propose to charge for a component
which demands more resources. That is, storage-oriented transactions should be
charged for state memory consumption, the computation-oriented transactions
should be charged for script execution, and all the others by the network load.
This can be formalized as follows:
\begin{equation}
    \operatorname{Fee}(tx) = \max\left(\alpha \cdot N_b(tx), \beta \cdot N_c(tx),
    S(state) \cdot \sum_i (B_i \cdot L_i) \right)\,.
    \label{eq:max}
\end{equation}
Here $\alpha$ and $\beta$ are the pricing coefficients, $N_b(tx)$ is transaction size 
which defines the network load, $N_c(tx)$ is the estimation of the computational 
cost of transaction, $S(state)$ is the cost of the storing one byte in the 
state for the unit of time~(a block), $L_i$ is the time for which the output $i$ is being stored 
in the state, and $B_i$ is its size in bytes.

Since the time for the data to reside in the state is usually unknown,
the third argument of $\max()$ in Eq.~\eqref{eq:max} cannot be deduced directly at
transaction submission time. For this purpose we introduce a notion of 
scheduled payments later in Section~\ref{sec:scheduled}. The third
argument in Eq.~\eqref{eq:max} becomes dominant over time. Starting from the moment $sT$ when the transaction happens, the
fee is increasing at a constant rate (see Fig.~\ref{fig:max_t}). The possible implementation of this 
algorithm is described in Section~\ref{sec:scheduled}.
\begin{figure}[h]
    \centering
    \begin{subfigure}[b]{.45\textwidth}
    \includestandalone[width=\textwidth]{figures/subsid}
    \caption{Transaction cost as a function of the output existence time.
        \newline
        \label{fig:max_t}}
    \end{subfigure}
    ~
    \begin{subfigure}[b]{.45\textwidth}
        \includestandalone[width=\textwidth]{figures/max_est}
        \caption{Space of transactions split by
            Eq.~\eqref{eq:max} into the subregions of the dominant fees.
            \label{fig:max}}
        \end{subfigure}
        \caption{Fee differentiation by resource consumption}
\end{figure}

The remaining questions here are the following. First, what are the guiding principles for
choosing $\alpha$, $\beta$ and $S(\cdot)$? Second, how can one estimate $N_c(tx)$?
For Turing--complete languages second can only be solved by executing
script in general case. The problem is known as the worst case execution
time~\cite{Wilhelm2008}, and is left beyond the scope of the paper. The
first question is answered below.

\subsection{Choice of the relative values of $\alpha$, $\beta$, $S(state)$}

Assume for now, that for every transaction we know the duration of outputs'
storage in the state. We will overcome this difficulty later. Based on
Eq.~\eqref{eq:max}, one can introduce the space of transactions, which
is three--dimensional in our case --- every transaction is defined by three
numbers: $N_b(tx)$, $N_c(tx)$, $N_s(state,tx) = \sum_i (B_i \cdot L_i)$.
Eq.~\eqref{eq:max} divides
this space into three regions: network--oriented transactions, space--oriented transactions, 
and computation--oriented transactions (see Figure~\ref{fig:max}). The splitting
is governed by the direction of vector $\vec{n}$ which defines the line $\alpha
N_b=\beta N_c=S(state)N_s$.
Varying the coefficients $\alpha$ and $\beta$, one can change the direction of
$\vec{n}$ adjusting the formal fee prescription to the sensible values.

\subsection{Choice of $S(state)$}

The simplest way of assigning the $S(state)$ value is by making it constant.
%That is, specifying the price of storage of 1 byte of data per one day. 
However, %as it is shown in the Appendix~\ref{apx:statesize} \dnote{do we keep appendix?}, 
this does not fully solve the problem of limiting the state size. What is being controlled in this case,
that is the rate at which the data is being submitted, but not the state size
itself. One could also manually define the maximal size of the state for the
network. This solution, in turn, has its own caveats. For example, once the
state is kept (almost) full by the participants, it can be (almost) impossible
to submit the transaction increasing the state size.  The time till it becomes
possible is hardly predictable. 

The desired properties of the current state size could be formulated as follows:
it should be predictable, stable, and below some externally given value (an upper
bound on state size, being unique for the whole network). 

Another natural question arising is whether the rigid state size restriction is
necessary?  It is easy to imagine the situation where the formal possibility of
exceeding the state size upper bound is still present, but hardly ever being used. For example,
if one wants to constrain the state size to 10MB, the possible solution is to
set normal price for submitting data to store if the state size after submission
is below 10MB, but some astronomical price for the luxury of storage above 10MB.
So, formally it will be possible, but in fact, hardly ever used, with every
usage bringing significant profit to the miners. The generalization of this idea is
to form the explicit dependence of price on the state load (it will referred to
as ``pricing curve''). A good pricing curve should provide at least one stable
equilibrium of the state size; the minimal dependence on initial conditions (if
possible), and high rewards for miners. The latter could serve as good
optimization parameter. Extreme cases are zero price with huge data submission and
miners get nothing; and infinite price with zero data submission and miners get
nothing. As usual, the maximal outcome is in between.  The pricing policies
described above are two particular cases of pricing curve (see
Figure~\ref{fig:steps}). That is, we assume that the price of data storage in the
state $S(state)$ varies with the current state load $x=|s|$. 
\begin{figure}[h]
    \hfill 
    \begin{tikzpicture} 
        \draw[thick,-stealth] (0,0) -- (0,3) node [right]{$S(state)$}; 
        \draw[thick,-stealth] (0,0) -- (5,0) node [above]{load}; 
        \draw[very thick] (0,0.3) -- (3,0.3) -- (3,3); 
        \draw[dotted] (3,1) -- (3,0) node[below]{$10$MB}; 
    \end{tikzpicture} 
    \hfill 
    \begin{tikzpicture} 
        \draw[thick,-stealth] (0,0) -- (0,3) node [right]{$S(state)$}; 
        \draw[thick,-stealth] (0,0) -- (5,0) node [above]{load}; 
        \draw[very thick] (0,0.3) -- (3,0.3) -- (3,2) -- (5,2); 
        \draw[dotted] (3,1) -- (3,0) node[below]{$10$MB}; 
    \end{tikzpicture} 
    \hfill 
    \caption{ 
        Examples of pricing curves: rigid state size restriction (left) and 
        overflow fees (right, see text). The value of $10$MB is taken 
        arbitrarily.  
    } 
    \label{fig:steps}
\end{figure} 

Note that the pricing curve is defined by a small number of parameters and
to be the same for all the network. To impose an upper bound on the state
size, one can choose the pricing curve formally going to infinity at some finite
state size. The rigid boundary can be provided by divergence higher than
$1/(x_{max}-x)$. One can also try to estimate the optimal state size for a given
differentiable pricing curve. 
%In the continual with the assumptions described in the appendix, 
The data submission rate $N(S(x))$ is fully defined by the
current storage price $S(x)$.  Rewards rate obtained by the miners for stable
state size at price $S$ per unit time is given by $y = S \cdot N(S)$. An example is provided in
Figure~\ref{fig:rewards}. First, it provides a possible method of measuring
explicit form of the function $N(S)$ in the model: one has to set up the price,
and observe the static rewards. Second, one may wonder about the price $S^*$,
optimal for the miners in terms of rewards. Obviously, it satisfies
$N(S^*)+S^* \cdot N'(S^*)=0$, where prime %what is "prime" here?
is derivative with respect to price. As
usual, the optimal price here does not depend on the pricing policy, but rather
the implicit property of the network. Having the price varying freely can be
considered beneficial both for miners and for network as a whole, since it
allows the first ones to optimize signing strategy, and herewith the
state size is automatically adjusted to the relatively predictable level
$S^{-1} (S^*)$.

\begin{figure}[H]
    \includestandalone[width=.9\textwidth]{figures/rewards}
    \caption{
        \label{fig:rewards} Example of the rewards curve.
    }
\end{figure}

\section{Scheduled Payments}
\label{sec:scheduled}

In this section we propose a concrete method to charge for state bytes consumed~(or released). There is a couple of possible options for that. A user, for example, may specify lifetime for a coin during its creation and pay for it in advance, this is not very convenient for him though. Another option is to charge when coin is spent, or allow to spend a coin~(by anyone, presumably, a miner) when its value is overweighted by the state fee. As a drawback, if coin is associated with a big value, it could live for very long time, maybe without a reason.

We propose more convenient method of charging; we name it {\em scheduled payments}. In this scheme a user must set special predefined script for a coin~(otherwise a transaction and also a block containing it are invalid), which contains a user-specific logic~(we call it a {\em regular script}) and a spending condition which allows anyone~(presumably, a miner) to create a transaction claiming this output, necessarily creating a coin with the same guarding condition and a value not less than original minus a state fee. These two parts~(regular script and a fee charging condition) are connected by using the $\lor$ conjecture. We assume that $\alpha$ and $\beta$ are fixed. We also assume that subsidized period $sT$ is to be stored along with the coin by each validating node. Then a guarding script for the coin would be like:

\begin{align}
\begin{split}
&(regular\_script) \lor \\
&(height > (out.height + sT) \land (out.value \le S_c \cdot B \cdot sT \lor \\  
&\qquad tx.has\_output(value = out.value - S_c \cdot B \cdot sT, script = out.script))),
\end{split}
\end{align}
where $height$ is a height of a block which contains a spending transaction; $out.height$ is a height when the output was created; $out.height$ and $out.script$ contain value and spending script of the output, respectively; $tx.has\_output()$ checks whether a spending transaction has an output with conditions given as the predicate arguments, and $S_c$ is the value of $S(state)$ when the coin was created. As in the Section~\ref{sec:algorithm}, the $B$ constant is the output size.


\section{Evaluation}
\label{sec:evaluation}

In this section we experimentally study what could be the real-world
ratio between the pricing coefficients $\alpha, \beta, S(state)$. To extract the realistic 
values, and to verify the validity of the described transaction classification, the
data from the Ethereum network is taken. We consider Ethereum a good example,
since all the three fee components are present in this cryptocurrency. The network load parameter $N_b(tx)$ is simply a transaction size; the state
load $\Delta(tx)$ can be deduced from the blockchain by extracting SSTORE
and CREATE operations from the transaction $tx$~\footnote{Information on these operations can be found in the Ethereum Yellow Paper~\cite{ethyp}.}. To determine the
computational load $N_c(tx)$, we count Ethereum gas consumed by the transaction processing minus its storage cost and the so-called base cost, which is proportional to the transaction size.

\begin{figure}[h]
    \centering
    \begin{subfigure}[b]{0.48\textwidth}
        \includegraphics[width=\textwidth]{figures/txs3d}
        \caption{}
        \label{fig:a}
    \end{subfigure}
    \begin{subfigure}[b]{0.48\textwidth}
        \includegraphics[width=\textwidth]{figures/txs-size-computation}
        \caption{}
        \label{fig:b}
    \end{subfigure}

    \begin{subfigure}[b]{0.48\textwidth}
        \includegraphics[width=\textwidth]{figures/txs-size-storage}
        \caption{}
        \label{fig:c}
    \end{subfigure}
    \begin{subfigure}[b]{0.48\textwidth}
        \includegraphics[width=\textwidth]{figures/txs-storage-computation}
        \caption{}
        \label{fig:d}
    \end{subfigure}

    \caption{Ethereum transactions classification by resource consumption}
    \label{fig:eth}
\end{figure}

The results of processing first $2\cdot10^6$ blocks in Ethereum network are presented in Figure~\ref{fig:eth}.
Each point corresponds to a transaction.
One can notice that parts of the distribution in Figure~\ref{fig:eth} extend
along the coordinate axis; these are the transactions which can be
unambiguously distinguished by their type of the resource consumption. Their
presence confirms our expectations on the nature of resource consumption, and
serves as a justification of the proposed classification scheme. The space of
transactions is split into three parts by the aforementioned vector $\mathbf{n}$
with the endpoint at the first momentum of the transactions with at least 2
non-zero components. %, and thus not evident affiliation to a concrete type.

Another parameter of interest is a storage object lifetime. Associating it directly
with smart contract data lifetime is weakly relevant to our scheme as the users are not incentivized to remove data from the state earlier rather than later. Thus we consider the delay between the data submission and a first request to be the reliable parameter reflecting the needs of the users.
Analysis of Ethereum blockchain shows that in a lot of cases data stored in the world state
is touched by other transactions in the same block or few blocks after creation.
We filter out such cases as they do not show using blockchain as a storage. Excluding such short-lived data from our analysis
we estimate that average lifetime of a data object in Ethereum is 23,731 blocks~(or about 4 days
considering 15 seconds average delay between blocks).

This gives the following estimation on the ratio between the pricing coefficients for the expected state size:
\begin{align}
\begin{split}
&\frac{\alpha}{\beta} \approx 7.7\cdot10^{-3} \\
&\frac{\alpha}{S} \approx 6.7\cdot10^{-4},
\end{split}
\end{align}
where $S$ is the cost of the storage of byte of output in the state for one block, which does
not depend on state size in Ethereum. The estimations are quite approximate, while changing
them does not affect fees for most of transactions unambiguously attributed by a concrete type of the resource consumption.

\section{Concluding remarks}
\label{sec:conslusion}

Blockchain technology relies on miners, that safeguard the integrity of the blockchain
in exchange for a revenue, that usually consist of two parts: block reward and transaction fees.
Transaction fees are useful to limit miners resource usage and prevent spam.

While in most of cryptocurrencies a transaction fee is addressed as an atomic concept,
in this paper we have shown that it is reasonable to introduce the 
components of a fee associated with resources utilized: network, computation or storage.
The analysis of Ethereum blockchain shows that transactions in such a three-dimensional space are
distributed close to one of the axes, allowing us to
unambiguously classify transactions by consumed resource.

Storage part of the fee has already been discussed in literature as a
necessary tool to limit miners storage consumption~\cite{Moeser2015,reyzin2016improving}.
This fee component is required to make the state size more predictable, but its implementation is
challenging since the output lifetime is not known at the time when the transaction is
created. In current paper we have described the concrete method to charge for state bytes consumed
that can be fully implemented on the script level.

Besides limiting the size of the state, storage fee provides valuable side effects.
In particular, it provides a way to return coins with lost keys into circulation.
Although necessity of coin recirculation is still an open question, it
has been widely discussed in literature (e.g.~\cite{gjermundrod2014recirculating,gjermundrod2016going})
in connection with the prevention of the deflation, which may eventually occur in cryptocurrencies with fixed supply. Enforced coin recirculation has been implemented in some cryptocurrencies~\cite{freicoin}.

Another important side effect of the storage fee is that it provides additional rewards 
for miners. Even when all coins are emitted and fixed block reward goes to zero,
storage fee will provide stable rewards for miners, which do not depend on user transactions included into block.
%that only depends on the fact of block creation,
This will make destructive mining strategies described in~\cite{carlsten2016instability} less profitable.

With these factors taken into account, the ready-to-implement system is provided,
which is believed to solve the problem of uncontrollable state growth. It
bears some valuable side effects by the same means, while preserving currently
existing methods for transaction fees and code execution costs.


%In the present work we addressed the problem of growth of the state size for the
%blockchain systems. Proposed solution --- spacetime fees --- relies on the
%economical factors rather than on solely the protocol itself. The idea of
%spacetime fees led us to the necessity of classification of the transactions by
%the type of resource consumption, and the simplest fee realization by
%$\max$-estimate caused the introduction of chain-based {\it scheduled payments}.
%It is shown that the described scheme can be fully implemented on the script
%level. The analysis of Ethereum blockchain evinced self-consistency of our
%treatment, and gave the reference parameters for possible implementation. We
%argue that the proposed solution makes the state size more predictable, which is
%crucial for decentrslization, and it can be achieved even without rigid
%restriction on the maximal state size. A valuable side effect of the spacetime
%fees is recirculation of the lost UTXOs in a long-run. With these factors taken
%into account, the ready-to-implement system is provided, which is believed to
%slove (at least partially) the problem of uncontrollable state growth, and
%recirculation of UTXOs by the same means, while preserving currently existing
%methods for transaction fees and code execution costs.

\bibliographystyle{splncs03}
\bibliography{sources.bib}

%\appendix
%
%\section{State size dynamics}
%\label{apx:statesize}
%For the primary analysis we assume that participants act honestly: they submit
%data if they need to do so and it is affordable; they do not in the opposite
%case. For the sake of simplicity, suppose that the time of storage of data
%block is fixed, and equal to $T$. In order to reduce the discrete stochastic
%model to continuous deterministic one, assume that the number of participants
%is large, and the typical size of the submitted data chunk is much less than
%characteristic pricing curve variation scale. The amount of data submitted to
%the state is changing continuously. At every given moment of time the users
%submit the data at some rate (say, MB/s) $f$, which is defined by the current
%price (participants submit more when cheap, and less when expensive). The
%current price is fully determined by the current state load $x$. After time
%interval $T$ data is erased from the state. The data is written into the state
%at rate $f(x(t))$, and erased at rate $f(x(t-T))$. Under these assumptions,
%the evolution of the state load is defined by the following equation:
%\begin{equation}
%    \frac{dx}{dt} = f(x(t))-f(x(t-T))\,.
%    \label{eq:dde0}
%\end{equation}
%If one measures time in the data block TTL $T$, the equation takes the form
%\begin{equation}
%    \dot{x} = f(x(t))-f(x(t-1))\,.
%    \label{eq:dde1}
%\end{equation}
%The equations of this type are called delay differential equations (DDE), and have
%been studied widely for the vast amount of control problems.
%
%Now few words about the function $f$, and how to convert it to the miner's
%income. What participants know is the pricing curve $S(x)$. For every value $P$
%of this function there is amount of people $N(S)$ who want to submit data and
%can afford it at time interval $T$. The function $N(S)$ is non-increasing, and
%going to zero for sufficiently large $S$ (every participant has maximal price he
%is ready to pay, and will also try to submit something if current price is
%lower; for sufficiently large price no one is ready to pay). In these notations
%$f(x)=N(S(x))$ (see inset on Figure~\ref{fig:rewards}), and the profit rate which
%miners get from current submissions is $y(t) = S(x(t))f(x(t))$.
%
%\subsection{Constant storage price}
%\subsection{State-dependent storage price}
%\begin{figure}
%    \includestandalone[width=\textwidth]{figures/dynamics}
%    \caption{
%        \label{fig:dynamics} Dynamics of the state size for various initial
%        rates. The numbers above the curves are the miners reward rates with
%        respect to maximal possible stationary rewards.
%    }
%\end{figure}

\end{document}
